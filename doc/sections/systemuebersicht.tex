\documentclass[../main.tex]{subfiles}
\graphicspath{{\subfix{../images/}}}
\begin{document}
\section{Systemübersicht}

\subsection{Hardware}
Unser Zeiterfassungssystem besteht aus zwei Hardware basierten Hauptkomponenten. Zum einen aus einem RFID RC522 NFC Reader, der mit einem Node MCU vom Typ ESP8266 Mod 12-F kommuniziert. Zum anderen NFC Chips, welche in diversen NFC Tags oder Karten verbaut sind.

\subsubsection{ESP8266 und RC522}
Wir haben uns für den ESP8266 entschieden, da der IOT-Oktopus keine freien Schnittstellen für den RC522 geboten hat, ohne dabei gewisse Schnittstellen doppelt zu belegen, was ein unvorhersehbares Verhalten der am Oktopus verbauten LED oder des RFID Readers zufolge hatte.

\subsubsection{Zusätzliche Komponenten}
Um die passende Kommunikation zwischen ESP8266 und RFID RC522 NFC Reader zu gewährleisten wurde eine Verbindungsplatine designt und im KI Labor mithilfe des dort vorhandenen Platinendruckers gefertigt. Zudem wird das System durch eine 3D gedruckte Hülle vor leichten Verschmutzungen geschützt und zusätzlich noch etwas ansehnlicher gestaltet. 



\subsection{Firmware}
Die Firmware des ESP8266 steuert das Verhalten der Hardware. In dieser findet die Verbindung des ESPs mit einem geeigneten Router zur Herstellung einer aktiven Internetverbindung, der Zugriff auf einen geeigneten NTP Server und die Verarbeitung der vom RFID RC522 NFC Readers erhaltenen Signale statt. Essenzielle Daten, wie der API-Key oder die Verbindungsdaten zum gewünschten Router stehen hierbei nicht im Code selbst, sondern in den1 separat ausgelagerten Header-Dateien \say{api.h} und \say{credentials.h}.

\subsubsection{Datenerfassung}
Erhält der ESP8266 ein Signal vom RFID RC522, so wird mit diesem Signal die ID des NFC Tags übermittelt. Anschließend wird ein Update der Zeit vom NTP Server angefordert. Folglich werden die beiden erhaltenen Daten \say{ID} und \say{Zeit} zusammen mit dem API-Key in eine temporäre JSON-Datei verpackt und an den in der \say{api.h} eingetragenen Server übermittelt. 

\subsection{Server}

Der Server läuft über einen Docker Container und kann somit flexibel und überall gehostet werden. Die verwendeten Ports müssten aber noch manuell freigegeben werden, damit der Node MCU darauf zu greifen kann. Sonst kann auf die Software zu Testzwecken via Kommandozeile gestartet werden. Dabei ist wichtig, dass die Adresse wo der Server verfügbar ist auf 0.0.0.0 gesetzt wird, damit er auch im Ganzen Netz verfügbar ist.

\subsubsection{Aufbau}

Der Server besteht aus drei wichtigen Komponenten, sie bilden die Kernfunktionalität der Software

\begin{itemize}
    \item Pug.py (Rendert das Frontend)
    \item Flask (API)
    \item SQLite (Datenbank)
\end{itemize}

\end{document}